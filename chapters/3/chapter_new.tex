\startchapter{The Large Hadron Collider and the ATLAS Experiment}
\label{chapter:lhcatlas}

The \textbf{Large Hadron Collider (LHC)} is the world's largest and most energetic proton-proton collider. It forms a 27 km circular ring beneath Switzerland and France, with its origin at the \textbf{Conseil Européen pour Recherche Nucléaire (CERN)} in Geneva. It began operation in 2008, and since then has collided over $10^{15}$ particles. The collider houses four major experiments: ALICE, ATLAS, CMS, and LHCb, along with several smaller projects. ALICE (A Large Ion Collider Experiment) studies heavy ion collisions, while LHCb (Large Hardon Collider beauty) specializes in studying the physics of the bottom quark. \textbf{ATLAS (A Toroidal LHC ApparatuS)} and CMS (Compact Muon Solenoid) are both general purpose experiments designed to study a wide range of interactions resulting from high-energy proton-proton collisions. This work uses data collected by the ATLAS experiment, and a description of the LHC accelerator and ATLAS detector follow in Sections \ref{section:lhc} and \ref{section:atlas} respectively.

\section{The Large Hadron Collider}
\label{section:lhc}
The LHC

\section{The ATLAS Experiment}
\label{section:atlas}
The Atlas Experiment
\subsection{Inner Tracking Detector}
The inner tracking detector
\subsection{Calorimeters}
The calorimeters
\subsection{Muon Spectrometers}
The muon detectors
\subsection{Trigger and Data Acquisition}
The trigger and DAQ

% \input chapters/3/sec_latexhelp
