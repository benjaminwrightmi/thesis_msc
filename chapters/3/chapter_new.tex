\startchapter{The Large Hadron Collider and the ATLAS Experiment}
\label{chapter:lhcatlas}

The \textbf{Large Hadron Collider (LHC)} is the world's largest and most energetic proton-proton collider. It forms a 27 km circular ring beneath Switzerland and France, with its origin at the \textbf{Conseil Européen pour Recherche Nucléaire (CERN)} in Geneva. It began operation in 2008, and since then has collided over $10^{15}$ particles. The collider houses four major experiments: ALICE, ATLAS, CMS, and LHCb, along with several smaller projects. ALICE (A Large Ion Collider Experiment) studies heavy ion collisions, while LHCb (Large Hardon Collider beauty) specializes in studying the physics of the bottom quark. \textbf{ATLAS (A Toroidal LHC ApparatuS)} and CMS (Compact Muon Solenoid) are both general purpose experiments designed to study a wide range of interactions resulting from high-energy proton-proton collisions. This work uses data collected by the ATLAS experiment, and a description of the LHC accelerator and ATLAS detector follow in Sections \ref{section:lhc} and \ref{section:atlas} respectively.

\section{The Large Hadron Collider}
\label{section:lhc}
Built between 1998 and 2008, the LHC began colliding protons in 2010 at a center-of-mass energy of 7 TeV, and today most recently in 2018 collisions occured at 13 TeV. Successfully colliding particles at this energy is an immense technical challenge which is achieved by the many technologies of the LHC and its accelerator complex.

In order to reach a beam energy of 6.5 TeV, protons are slowly stepped up through a chain of accelerators before reaching the LHC. They begin their journey as hydrogen atoms, stripped of their electrons before beign accelerated to an energy of 50 MeV by the Linac2 linear accelerator. Following this the Proton Synchrotron Booster (PSB) accelerates them to 1.4 GeV, passing them to the Proton Synchrotron (PS) and then the Super Proton Synchrotron to be accelerated to 25 and then 450 GeV. Finally, beams are split into clockwise and counterclockwise directions and injected into the LHC where they are accelerated to theiir final 6.5 TeV energy.

The acceleration of the protons is achieved by radio-frequency cavities. These cavities contain a resonant electro-magnetic field oscillating at 400 MHz, which is applied to particles passing through. The LHC contains 8 cavities per beam, with each providing a maximum of 2 MV of potential, so each proton can recieve up to 16 MeV of energy per lap. As a result it takes millons of laps over a period of around 20 minutes for a proton injected at 450 GeV to reach its collision energy of 6.5 TeV.  These RF cavities also serve to keep each beam in bunches of $1.15 \times 10^{11}$ protons spaced at intervals of just 25 ns.

The crown jewel of LHC technology is its magnets. 1,232 superconducting NbTi dipole magnets kept at 1.9 K, each spanning 14.3 m and weighinh 35 tonnes, create an 8.3 Tesla magnetic. This field lies perpendicular to the beam path, bending it to its desired route. The bending dipole magnets are complemented by 392 quadrupole magnets that focus the beams to a small aperture, and many higher-order multipole magnets which provide small beam corrections.



\section{The ATLAS Experiment}
\label{section:atlas}
The Atlas Experiment
\subsection{Inner Tracking Detector}
The inner tracking detector
\subsection{Calorimeters}
The calorimeters
\subsection{Muon Spectrometers}
The muon detectors
\subsection{Trigger and Data Acquisition}
The trigger and DAQ

% \input chapters/3/sec_latexhelp
