\startchapter{Conclusions}
\label{concl}
This thesis has presented the design of the most sensitive ``merged" signal region in the search for dark matter produced with a new heavy vector boson, $Z'$, and a scalar ``dark Higgs" boson, $s$, decaying to a pair of $W$ bosons with a semileptonic ($q\bar{q}\ell\nu$) final state. It has also presented the design of control regions to constrain the \ttbar SM background, as well as the important supporting analysis work.  The search uses data collected by the ATLAS experiment in 2015-2018, from proton-proton collisions at a center-of-mass energy of 13 TeV.

The search for the dark Higgs model \cite{Hunting} is motivated by the desire to detect and characterize dark matter, as well as the need for some mechanism to generate the mass of particles in the dark sector. The SM is an extremely successful theory of the fundamental particles and their interactions, however astrophysical observations demand the existence of dark matter beyond the current confines of the model. The dark Higgs model introduces dark matter as a Majorana fermion, with a mass generated by the Higgs mechanism from a new Higgs field with an associated dark Higgs boson $s$. Through the mixing of the SM Higgs boson and the $s$, the $s$ can interact with all massive SM particles and can thus decay to SM final states that are visible in the ATLAS detector.

To search for this signature, we have analyzed proton-proton collion events measured in the ATLAS detector. We have begun by defining a set of phyiscs objects representing particles and groups of particles in the ATLAS detector. We have then used those definitions to reconstruct and characterize events in the ATLAS detector. Using the reconstructed events we have defined a signal region consisting of an optimized set of event-by-event criteria that, based on MC simulated data, are expected to select many events containing signal (mono-$s$ decay) processes and few SM background events.

We have subsequently defined the control regions as regions near to, but not overlapping with, the signal regions, which are statistically dominated by the \ttbar background process. We compared the MC simulated SM background to the measured ATLAS data in these regions to gain information about the accuracy of our MC predictions and constrain our predictions in the signal region. Based on the SM and signal predictions given by MC in the signal region, and the constraints derived in the control regions, we have finally calculated and presented the expected sensitivity of the signal regions to the signal model, which represents our ability to make conclusions about its existence using observations in those regions.

Using the designed regions described in this thesis, the search for the semileptonic $s$ decay will be able to substantially improve upon the sensitivity to the dark Higgs model achieved by prior searches in other decay channels.
