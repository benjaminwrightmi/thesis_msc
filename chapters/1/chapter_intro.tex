\startfirstchapter{Introduction}
\label{chapter:introduction}

Developed throughout the latter half of the 20th century, the \textbf{Standard Model of Particle Physics (SM)} \cite{peskin,schwartz,pdg_rev,Griffiths} is the most successful theory describing the fundamental particles and their interactions. It describes matter as consisting of \textbf{quarks} (which make up protons and neutrons) and \textbf{leptons} (such as electrons), which interact via three fundamental forces (the electromagnetic force, the strong force, and the weak force) via the exchange of mediating \textbf{bosons}. The \textbf{Higgs boson}, the final SM particle to be discovered, generates the mass of the other particles via the ``Higgs mechanism".  The predictions of the Standard Model have been extremely rigorously tested experimentally, and in almost all confirmed cases, have held true. There are, however, some areas in which the Standard Model falls short. For example, it does not include a description of the fourth fundamental force, gravity, nor does it explain the existence of dark matter, an invisible form of matter which is measured via astrophysical observations.

One extension to the SM proposed by theorists is the ``dark Higgs" model \cite{Hunting}. This proposes the existence of three new particles:
\begin{itemize}
\item a dark matter fermion $\chi$,
\item a heavy mediator boson $Z'$, and
\item a \textbf{dark Higgs boson} $s$.
\end{itemize}
In this model, the dark Higgs boson generates the masses of the $Z'$ and $\chi$ particles, and each of the three particles in this new ``dark sector" can interact with their two counterparts. The proposed dark sector interacts with the SM via the coupling of the $Z'$ to quarks and the mixing of the $s$ with the SM Higgs boson, which allows the $s$ to interact with SM particles.

In order to probe the limits of the SM, and to search for evidence of new phenomena such as the dark Higgs model, particle accelerators such as the \textbf{Large Hadron Collider (LHC)} collide protons at high energies. \textbf{A Toroidal LHC Apparatus (ATLAS)} experiment, one of the four main particle detectors at the LHC, detects the products of the proton-proton collisions, and measures their positions, energies, and momenta as they move through the detector. The collected data is then analyzed event-by-event to search for theorized signatures of new phyiscs.

One technique to perform an analysis, which will be discussed in this thesis, is the ``cut and count" method. We use this method to search for evidence of the dark Higgs signal model. We use Monte Carlo (MC) simulations to generate SM background and dark Higgs signal physics processes as they would occur from proton-proton collisions in the LHC, and also to simulate the behaviour of the resulting particles in the ATLAS detector. These simulations form the basis for us to predict what the ATLAS detector will measure with or without the existence of the signal model.

Such signal processes, however, would be rare, and searching for them in event displays among billions of proton-proton collisions would be impossible. Instead, using the cut and count method, we identify measurable properties of events that allow us to differentiate between events likely to be categorized as signal or background. We then cut our dataset down using these distinguishing properties to define a ``signal region", which we expect based on simulations to have a high concentration of signal events in the case where the dark Higgs model exists. We can then compare the expected distribution of events in this region from MC simulated data with measured ATLAS data and reach a conclusion about the likelihood that the signal model correctly describes nature.

This thesis focuses on the design of the most sensitive ``merged" signal region in the ATLAS analysis searching for the dark Higgs model, for the case where the $s$ produced decays to a pair of $W$ bosons, one of which decays to a pair of quarks and the other of which decays to a pair of leptons. It also discusses the design of a ``control region", which is used to constrain the SM expectation of one of the leading sources of background events.

Chapter 2 provides an overview of the Standard Model of particle physics, as well as the motivation for searching beyond its limits, and a description of the theoretical framework for the dark Higgs model. A description of the LHC and the ATLAS detector system are presented in Chapter 3. Chapter 4 describes analysis preparation work, including MC simulation and analysis object definition. The signal and control region design and definition are described in Chapter 5, while Chapter 6 presents the statistical framework used to evaluate the analysis sensitivity and the results. Finally, Chapter 7 provides a summary and conclusions.
